\documentclass[a4paper]{article}

\usepackage{palatino}
\usepackage{mathpazo}
\usepackage{microtype}
\usepackage{amssymb}
\usepackage{amsfonts}
\usepackage{amsmath}
\usepackage{amsthm}

\newtheorem{definition}{Definition}

\DeclareMathOperator{\semi}{;}
\begin{document}
\author{Alexander Weigl}
\title{Mini-SymEx -- Rules}
\maketitle

\newcommand\config[3]{\langle #1,\:  #2,\: #3 \rangle}

\begin{definition}
  A configuration of the symbolic execution is a tuple of a function
  $\sigma \colon Var \to Var$, a set of assumptions $A \subseteq Fml_\Sigma$,
  and the remaining program $R$.
  A configuration is   written as $\config{\sigma}AR$.
\end{definition}

\def\epsilon{\varepsilon}

Our rules use construct the single static assignment form (SSA) dynamical. Using
SSA reduces the size of verification, but makes the handling of branching a bit
more difficult.

Some small notations in the rules below:
\begin{itemize}
\item $\epsilon$ denotes the empty program.
\item $e[\sigma]$ denotes the application of the substitution $\sigma$ on the
  expression $e$. This is equivalent to the symbolical evaluation of the
  expression.
\item $\sigma[x \to y]$ is the update of the function at position $x$ to value
  $y$.

\item The function $fresh$ always returns a completely new unused variable.
\end{itemize}


\begin{equation}
  \label{ex:1}
  \frac%
  {
    \config{\sigma}{A}{S_1} \leadsto \config{\sigma'}{A'}{\epsilon}
  }%
  {
    \config\sigma{A}{S_1\semi S_2}
    \leadsto
    \config{\sigma'}{A'}{S_2}
  }\mathtt{comp}
\end{equation}

\begin{equation}
  \label{ex:1}
  \frac%
  {
    \quad%\config{\sigma}{S_1} \leadsto \config{\sigma'}{\epsilon}
  }%
  {
    \config\sigma A{\mathtt{assume}~e}
    \leadsto
    \config{\sigma'}{A \cup e[\sigma]}{\epsilon}
  }\mathtt{assume}
\end{equation}


\begin{equation}
  \label{ex:1}
  \frac%
  {
    A \implies e[\sigma]
    \quad%\config{\sigma}{S_1} \leadsto \config{\sigma'}{\epsilon}
  }%
  {
    \config\sigma A{\mathtt{assert}~e}
    \leadsto
    \config{\sigma'}{A \cup e[\sigma]}{\epsilon}
  }\mathtt{assert}
\end{equation}

Rule \texttt{assert} is the only rule which charges a verification condition, in
particular, $A \implies e[\sigma]$ must be valid. For SMT-solving, we apply the
deduction theorem and check $A \wedge \neg e[\sigma]$ for unsat.

\begin{equation}
  \label{ex:1}
  \frac%
  {
    a'=fresh()
    \quad
    A' = A \cup \{a' = e[\sigma]\}
    \quad
    \sigma' = \sigma[a \to a']
  }%
  {
    \config\sigma A{a~\mathtt{:=}~e}
    \leadsto
    \config{\sigma'}{A'}{\epsilon}
  }\mathtt{assign}
\end{equation}

\begin{equation}
  \label{ex:1}
  \frac%
  {
    a'=fresh()
    \quad
    \sigma' = \sigma[a \to a']
  }%
  {
    \config\sigma A{\mathtt{havoc}~a}
    \leadsto
    \config{\sigma'}{A}{\epsilon}
  }\mathtt{havoc}
\end{equation}

\begin{equation}
  \label{ex:1}
  \frac%
  {
    \begin{matrix}
      (T) & \config\sigma A{\mathtt{assume}~c\semi S_1} \leadsto \config{\sigma_T}{A_T}{\epsilon}\\
      (E) & \config\sigma A{\mathtt{assume}~\neg c\semi S_2}\leadsto \config{\sigma_E}{A_E}{\epsilon}\\
      %
      (M\sigma)& \sigma' = \sigma \cup \{ v \mapsto fresh() \mid \sigma_T[v] \neq \sigma_E[v] \}\\
      %
      (MA)& A' = A \cup \{  \sigma(v) = \mathtt{ite}(c[\sigma],  \sigma_T[v], \sigma_E[v])   \mid \sigma_T[v] \neq \sigma_E[v]  \}
    \end{matrix}
    \quad
    \quad
    A'
  }%
  {
    \config\sigma A{\mathtt{if}~(c)~S_1~\mathtt{else}~S_2}
    \leadsto
    \config{\sigma'}{A'}{\epsilon}
  }\mathtt{if}
\end{equation}

The cases (T) and (E) should be clear: (T) is the execution of then-branch of
the if-statement, and (E) for the else-branch, respectively. After both branches
are executed, we need to merge their state and assumption for conflicting
(re-assigned) variables. Every conflicting variable is assigned to a fresh
variable (M$\sigma$). In the assumption, we make a case distinction
($\mathtt{ite}$), whether to use value and assumptions of the then- or
else-branch.

\begin{equation}
  \label{ex:1}
  \frac%
  {
    \begin{matrix}
      (I) &\config{\sigma}{A%\cup\{c[\mu]\}}
      }{\mathtt{assert}~Inv}\\
      (P) &\config{\sigma}{A}{\mathtt{havoc}~E\semi\mathtt{assume}~Inv\land
        c\semi S\semi\mathtt{assert}~Inv}\\
      (T) &\config{\sigma}{A}{\mathtt{havoc}~E\semi\mathtt{assume}~Inv\land\neg
        c} \leadsto \config{\sigma'}{A'}{\mathtt{skip}}
    \end{matrix}
  }%
  {
    \config\sigma A{\mathtt{while}(c)~S}
    \leadsto
    \config{\sigma'}{A'}{\epsilon}
  }\mathtt{while}
\end{equation}

Note that $Inv$ is the invariant of the for the given loop, and $E$ is a set of
variables. This sets contains all variables which are written during in the loop
body $S$, and are therefore erased (havoc'd) when proving the preservation and
termination of the loop.

\begin{equation}
  \frac%
  {
    \begin{matrix}
      (C) &\config{id}{\{pre_f\}}{body_f\semi\mathtt{assert}~post}\\
      (U) &A \implies pre_f[\{ x_i \mapsto ... \mid 1\leq i \leq n \}]\\
      (T) & \sigma'=\sigma[v \mapsto fresh()]\\
      (T) & A' = A \cup\{post_f[ \mathtt{result} / \sigma(v)\} \\
    \end{matrix}
  }%
  {
    \config\sigma A{v \mathtt{:=} f(x_1, \ldots, x_n) }
    \leadsto
    \config{\sigma'}{A'}{\epsilon}
  }\mathtt{while}
\end{equation}

\end{document}
